% Related Work
\section{Related Work}
\label{RW}

%Introduce the representative existing algorithms and their weakness.

%Cite
%Algorithms:

%Deterministic:
%BlindDate 					#wang2015blinddate
%Disco 						#dutta2008practical
%Hello 						#sun2014hello
%Searchlight 					#bakht2012searchlight
%Talk More Listen Less 			#qiu2016talk
%Todis\&Hedis 					#chen2015heterogeneous
%U-Connect  					#kandhalu2010u

%probabilistic:
%Birthday 					#mcglynn2001birthday
%ALOHA-like09 				#vasudevan2009neighbor
%ALOHA-like11 	 			#you2011aloha
%PND 						#song2014probabilistic

%others
%Normal Distribution 			#wang2013gaussian
%Beacon&package 				#mcglynn2001birthday
%RSSI						#daiya2011experimental




Neighbor discovery problem has raised a great deal of attention of scholars \cite{sun2014energy}. 
A number of neighbor discovery methods have been proposed in the past decade.
Technically, these approaches can be classified into two categories, probabilistic and deterministic. 

In the deterministic methods \cite{dutta2008practical,kandhalu2010u,
bakht2012searchlight,sun2014hello,chen2015heterogeneous,
wang2015blinddate,qiu2016talk}, some mathematical techniques, such as
co-primality, quorum system, etc., are utilized to promote the discovery performance.
The deterministic methods holds an obvious advantage that they
can achieve neighbor discovery process within a bounded time latency.

Nevertheless, there exists some crucial weak points in the deterministic algorithm.
Firstly, Disco \cite{dutta2008practical} proposes a discovery protocol that
each node has a capability to send a beacon (one or a few bits) at both beginning and end of
an active slot, which is widely adopted by the later algorithms such as
SearchLight \cite{bakht2012searchlight}, BlindDate \cite{wang2015blinddate}
and Hello \cite{sun2014hello}, Nihao \cite{qiu2016talk}. It is quite an efficient way for two nodes to discover
each other within an ideal time latency. However,  they do not solve the collision 
issues when receiving packages from multiple neighbors. Furthermore, when they are 
extended for multiple nodes, only sending a beacon to discover the neighbors is 
totally insufficient. A node needs to send a complete package 
(some papers still call beacon) containing all its information,
otherwise the neighbor can not  identify which neighbor the beacon belongs to \cite{zhou2004impact}.
Thus a complete time slot is necessary for a node to transmit a package or listen to 
the channel to receive a package.

Another category is probabilistic methods \cite{mcglynn2001birthday,
vasudevan2009neighbor,you2011aloha,song2014probabilistic}. These
approaches utilize probability techniques to promote the randomness 
to discover the neighbors. Different from the deterministic algorithms, 
this kind of method shows an significant strength in the multiple nodes scenario. 
Relatively, probabilistic methods only present an expectation discovery latency and 
can not guarantee a latency bound in the worst case.
In addition, almost all the existing methods consider the network is fully-connected,
the topology of which is a complete graph. Deploying a fully-connected network 
in a large-scale area is technically impractical 
due to the limited sensing range of devices communication.
How far the other nodes can be detected as a neighbor for a mobile equipment  
depends on criterion such as the received signal strength \cite{daiya2011experimental}.
From our analysis and simulations, 
they hold a poor performance in the partially-connected networks, which is more 
practical in the reality world.

To the best of our knowledge, neither deterministic nor probabilistic methods are designed for
partially-connected networks. In this paper, we present a low-latency, energy-efficient 
neighbor discovery algorithm for partially-connected networks.

The proposed RDS-Alano and TP-Alano in this paper are a combination of both two categories,
and thus we compare our proposed algorithms with both deterministic and probabilistic methods. 
Particularly as mentioned above, the deterministic approaches need some adjustments 
when transferred to partially-connected networks. Details will be introduced in Section \ref{Evaluation}




