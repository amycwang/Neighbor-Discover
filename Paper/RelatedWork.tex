% Related Work
\section{Related Work}
\label{RW}

%Introduce the representative existing algorithms and their weakness.

%Cite
%Algorithms:

%Deterministic:
%BlindDate 					#wang2015blinddate
%Disco 						#dutta2008practical
%Hello 						#sun2014hello
%Searchlight 					#bakht2012searchlight
%Talk More Listen Less 			#qiu2016talk
%Todis\&Hedis 					#chen2015heterogeneous
%U-Connect  					#kandhalu2010u

%probabilistic:
%Birthday 					#mcglynn2001birthday
%ALOHA-like09 				#vasudevan2009neighbor
%ALOHA-like11 	 			#you2011aloha
%PND 						#song2014probabilistic

%others
%Normal Distribution 			#wang2013gaussian
%Beacon&package 				#mcglynn2001birthday
%RSSI						#daiya2011experimental
%Chinese Remainder Theorem		#ding1996chinese




Existing neighbor discovery algorithms can be technically classified into two categories, deterministic algorithms and probabilistic algorithms.

Deterministic algorithms adopt some mathematic tools to ensure discovery between every two neighbors. The first tool is called quorum system \cite{jiang2005quorum,luk1997two} : for any two intersected quorums, two neighboring nodes could choose any quorum in the system to design the discovery schedule. Hedis \cite{chen2015heterogeneous} is a typical one. Another important tool is co-primality where two co-prime numbers are chosen by the neighbors to design the discovery schedule, and they can discover each other within a bounded latency by the Chinese Remainder Theorem \cite{ding1996chinese}. Some representative algorithms are Disco \cite{dutta2008practical}, U-Connect \cite{kandhalu2010u}, and Todis \cite{chen2015heterogeneous}. These algorithms hold an obvious advantage that they can guarantee fast discovery for two nodes within a bounded latency. 

However, there exists some weak points in the deterministic algorithms when applying to large-scale networks.
U-Connect \cite{kandhalu2010u} only assumes two nodes turn on the radio at the same time and thus find each other. 
But in reality neighbor discovery is the process that a node receives a handshaking package from its neighbor successfully. 
Hedis and Todis \cite{chen2015heterogeneous} consider the transmitting and receiving roles.
However different from two-node scenario, collisions will happen when many nodes are transmitting simultaneously. 
Some deterministic algorithms propose a different transmission protocol.
For example, Disco~\cite{dutta2008practical} assumes that a node has a capability to send a beacon (one or a few bits) at both beginning and end of an active time slot, and the assumption is adopted in 
SearchLight~\cite{bakht2012searchlight}, Hello~\cite{sun2014hello}, and Nihao~\cite{qiu2016talk}.
Under the assumption, these algorithms could guarantee discovery between two nodes within an ideal latency. But sending a beacon may not work well in a large scale network since a node need to distinguish its neighbors by the received messages, which implies a transmitted package should contain the sender node's information\cite{zhou2004impact}. 


Another category is probabilistic algorithms \cite{mcglynn2001birthday,
vasudevan2009neighbor,you2011aloha,song2014probabilistic} which utilize probability techniques to promote the randomness
of discovering the neighbors. 
Birthday protocol \cite{mcglynn2001birthday} is one of the earliest algorithms that works on the birthday
paradox, i.e. the probability that two people have the same
birthday exceeds $\frac{1}{2}$ among $23$ people. Following that,
smarter probabilistic algorithms are proposed, such as Aloha-like \cite{vasudevan2009neighbor, you2011aloha}, PND \cite{song2014probabilistic}. Particularly, Aloha-like \cite{vasudevan2009neighbor} does not consider the energy consumption, which is later extended to an energy-restricted network by \cite{you2011aloha}.

Different from deterministic algorithms,
probabilistic algorithms show an significant strength for a network consisting of multiple nodes.
However, probabilistic algorithms only present an expectation discovery latency and they
can not guarantee a good latency bound.
In addition, most of the existing algorithms assume the network is a clique, which implies any two nodes are neighbors. 
This assumption can hardly depict a large-scale network due to the limited communication range of a node.
Some works adopt the received signal strength to decide how far a node can transmit~\cite{daiya2011experimental}, and in the protocol model, it is simplified that two nodes can communicate if their distance is no larger than a threshold.


%We implement some probabilistic protocols for a large-scale network and the results show that they have a poor performance (see Section \ref{Evaluation}).
%To the best of our knowledge, neither deterministic nor probabilistic algorithms have good performance when they are adopted in an energy-restricted large-scale network.





