% Related Work
\section{Related Work}
\label{RW}

%Introduce the representative existing algorithms and their weakness.

%Cite
%Algorithms:

%Deterministic:
%BlindDate 					#wang2015blinddate
%Disco 						#dutta2008practical
%Hello 						#sun2014hello
%Searchlight 					#bakht2012searchlight
%Talk More Listen Less 			#qiu2016talk
%Todis\&Hedis 					#chen2015heterogeneous
%U-Connect  					#kandhalu2010u

%probabilistic:
%Birthday 					#mcglynn2001birthday
%ALOHA-like09 				#vasudevan2009neighbor
%ALOHA-like11 	 			#you2011aloha
%PND 						#song2014probabilistic

%others
%Normal Distribution 			#wang2013gaussian
%Beacon&package 				#mcglynn2001birthday
%RSSI						#daiya2011experimental




Neighbor discovery has raised a great deal of attention in wireless networking area\cite{sun2014energy,XX,XX}, and a number of neighbor discovery methods have been proposed in the past decade.
Technically, these approaches can be classified into two categories roughly, deterministic algorithms and probabilistic based algorithms.

Deterministic methods utilize some mathematical techniques to promote discovery performance, such as co-primality, quorum systems, etc\cite{dutta2008practical,kandhalu2010u,
bakht2012searchlight,sun2014hello,chen2015heterogeneous,
wang2015blinddate,qiu2016talk}. These methods hold an obvious advantage that they can guarantee fast discovery for two nodes within a bounded latency. 
However, there exist some crucial weak points in the deterministic algorithm.
For example, Disco~\cite{dutta2008practical} assumes that a node has a capability to send a beacon (one or a few bits) at both beginning and end of an active time slot, and the assumption is widely adopted in many elegant algorithms such as
SearchLight~\cite{bakht2012searchlight}, BlindDate~\cite{wang2015blinddate}, Hello~\cite{sun2014hello}, TMLL~\cite{XX}, and Nihao~\cite{qiu2016talk}.
Under the assumption, these algorithms could guarantee discovery between two nodes within an ideal latency. But sending a beacon may not work well in a large scale network since a node need to distinguish its neighbors by the received messages, which implies a transmitted package should contain the sender node's information\cite{zhou2004impact}. Then, collision happens when many node are transmitting simultaneously. (??introduce some other algorithms which do not assume transmitting a beacon)

Another category is probabilistic methods \cite{mcglynn2001birthday,
vasudevan2009neighbor,you2011aloha,song2014probabilistic} which utilize probability techniques to promote the randomness
of discovering the neighbors. 
Different from deterministic algorithms,
this kind of method shows an significant strength for a network consisting of multiple nodes.
However, probabilistic methods only present an expectation discovery latency and they
can not guarantee a good latency bound.
In addition, most of the existing methods assume the network is fully-connected, which implies any two nodes are neighbors. 
This assumption can hardly depict a large-scale network due to the limited communication range of a node.
Some works adopt the received signal strength to decide how far a node can transmit~\cite{daiya2011experimental}, and in the protocol model, it is simplified that two nodes can communicate if their distance is no larger than a threshold.
We implement some probabilistic protocols for a large-scale network and the results show that they have a poor performance (see Section \ref{XX}).


To the best of our knowledge, neither deterministic nor probabilistic methods have good performance when they are adopted in an energy-restricted large-scale network.


(??comments: say more words about related algorithms, maybe one sentence for each protocol; add some algorithms for duty cycle based discovery since we also consider energy constraints)



