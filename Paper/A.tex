\section{Alano Algorithm for a Large Scale Network}
%\section{Large-Scale Networks}
\label{PCN}

%Expectation	#meester1996continuum

%background and instances
In a large scale network, nodes are not fully connected and communications may fail due to concurrent transmissions. 
When we do not consider the energy constraints of nodes, we propose a nearly optimal algorithm for a large scale network, which implies $\theta_i = 1$ for all node $u_i$.
Suppose the locations of nodes obey some distribution, we propose Alano algorithm and analyze its performance for two common used distribution (uniform distribution and Gaussian distribution).

\subsection{Alano Algorithm}
Suppose the locations of nodes obey some distribution and node $u_i$ is aware of its position coordinates $(x_i, y_i)$.
Then $u_i$ could compute its local density as follows \cite{meester1996continuum, wang2015connectivity}.

Denote the general density function as :
$$f(x,y)=
\begin{cases}
\varphi(x,y)& (x,y)\in D\\
0& (x,y)\notin D
\end{cases}$$
where $(x,y)$ is a position coordinate, and $D$ is the network covering area.

For each node $u_i$ with position $(x_i,y_i)$, the range of its neighbors' positions (denote as $R_i$) is formulated as:
$$
(x-x_i)^2+(y-y_i)^2 \leq \Delta^2
$$
where $\Delta$ is the communication range. Then, node $u_i$'s expected number of neighbors (denote as $\hat{n_i}$ is:
$$
\hat{n_i} = N\iint_{R_i} f(x,y)\,dx\,dy - 1.
$$

In a large scale network, we ignore the boundary area of the network and assume nodes in the network are of an enormous quantity. Then, the $\hat{n_i}$ can be rewritten as:
$$
\hat{n_i} = N\iint_{R_i} \varphi(x,y)\,dx\,dy.
$$

When the network covering area $D$ is much larger than the area $R_i$ of node $u_i$, 
\begin{equation}
\label{eqnNB}
\hat{n_i} \simeq N\pi r^2 \varphi(x,y).
\end{equation}


%Alano
We present \textbf{Alano}, a randomized algorithm for node $u_i$ in Alg. \ref{Alano}.
By computing the expected number of nodes, $u_i$ choose to be in transmitting state or listening state according to the generated probability on Line 2.


\begin{algorithm}
\caption{Alano Algorithm}
\label{Alano}
\begin{algorithmic}[1]
\STATE $\hat{n_i} = N\iint_{R_i} \varphi(x,y)\,dx\,dy$;
\STATE $p_t^i = \frac{1}{\hat{n_i}}$;
\WHILE {$True$}
	\STATE A random float $\epsilon \in (0,1)$;
   	 \IF{$\epsilon < p_t$}
    		\STATE Transmit a message containing node information of $u_i$;
	\ELSE
    		\STATE Listen on the channel and decode the node information if receive a message successfully;
	\ENDIF
\ENDWHILE
\end{algorithmic}
\end{algorithm}

In the following parts, we first consider a general situation that nodes in the network are uniform distributed. We derive a proof that the probability chosen in Alano is the optimal one and show that the discovery latency will not be much larger than its expectation. Then we analyze a more common situation that nodes obey Gaussian Distribution, we present an
approximation analysis that the discovery latency is not be much larger than that of uniform distribution.


\subsection{Analysis for Uniform Distribution}
\label{uniform}
%background
Uniform distribution is a basic one for deployment of wireless networks.
For instance, to monitor an unknown area, many sensors are deployed uniformly to sense information, such as temperature and humidity\cite{XX}. The nodes are evenly deployed and the density function is:
$$f(x)=
\begin{cases}
\frac{1}{A}& (x,y)\in D\\
0& (x,y)\notin D
\end{cases}$$
where $A$ is the area of $D$.

By Eqn. (\ref{eqnNB}), $u_i$'s expected number is $\hat{n} = \frac{N\pi r^2}{A}$ and the probability in Line 2 is set as $p_t = \frac{1}{\hat{n}}=\frac{A}{N\pi r^2}$.

From Alg. \ref{Alano}, the probability that node $u_i$ discovers a specific neighbor (such as $u_j$) successfully in a time slot (denote as $p_s$) is:
$$
p_s = p_t{(1-p_t)}^{\hat{n}-1}.
$$
In order to compute the maximum probability to discover a node, we compute the differential function of $p_s$ as:
$$
p_s' = {(1-p_t)}^{\hat{n}-1}-(\hat{n}-1)p_t{(1-p_t)}^{\hat{n}-2}.
$$
It is easy to verify that when $p_t=\frac{1}{\hat{n}}$, $p_s$ gets the maximum value:
$$p_s = \frac{1}{\hat{n}}{(1-\frac{1}{\hat{n}})}^{\hat{n}-1} \approx \frac{1}{\hat{n}e}.$$

Therefore, the probability chosen in Alano algorithm for transmitting is optimal.

We analyze the expected discovery latency for a node $u_i$ to discover all neighbors. Denote $W_j$ as a random variable representing the cost number of the time slots to discover a new neighbor after $j-1$ neighbors have been discovered. It is easy to check that $W_j$ follows Geometric distribution with parameter $p(j)$ where $p(j)=(\hat{n}-j+1)p_s$\cite{XX}. The expectation
of $W_j$ can be computed as:
$$
E[W_j]=\frac{1}{p(j)}=\frac{1}{(\hat{n}-j+1)p_s}.
$$

Then, the expectation discovery latency of node $u_i$ is:
$$
E[W_j] = \sum_{j=1}^{\hat{n}}\frac{1}{p_s}H_n \approx ne(lnn + \Theta(1)) = \Theta(nlnn).
$$
where $H_n$ is the $n$-th Harmonic number $H_n = lnn + \Theta(1)$.


The expected discovery latency is bounded by $O(n\ln n)$, and we show that the discovery latency can not be much larger than the expectation.

%To be modified later
%%
If $W_i$ is already computed, the value of $W_j$ will not be affected for $i<j$. That is, for $i\ne j$, $W_i$ and $W_j$ are independent and they satisfy $P(W_j=w_j|W_i=w_i)=P(W_j=w_j)$. Since $W_j$ follows Geometric distribution, $Var[W_j]=\frac{1-p_j}{p_j^2}$, and the variance of discovery latency (denoted as $W$???what is W) is computed as:
\begin{displaymath}
\begin{split}
 Var[W] %& =Var[\sum_{j=1}^{p_nN}W_j]=\sum_{j=1}^{p_nN}Var[W_j]+\sum_{j\ne k}Cov[W_j,W_k] \\
 =\sum_{j=1}^{n}Var[W_j]
%=\frac{1}{p_{suc}^2}\sum_{j=1}^{p_nN}\frac{1}{j^2}-\frac{1}{p_{suc}}\sum_{j=1}^{p_nN}\frac{1}{j} \\
 \le\frac{\pi^2}{6p_{suc}^2}-\frac{H_n}{p_{suc}}.
\end{split}
\end{displaymath}

By \emph{Chebyshev's inequality}, the probability that the discovery latency is $2$ times larger than the expectation is
\begin{displaymath}
\begin{split}
P[W\ge2E[W]]%=P[|W-E[W]|\ge E[W]]
\le\frac{Var[W]}{E[W]^2}
%&\le\frac{\frac{\pi^2}{6p_{suc}^2}-\frac{H_{p_nN}}{p_{suc}}}{\frac{H_{p_nN}^2}{p_{suc}^2}}=
\le\frac{\pi^2}{6H_{n}^2}-\frac{p_{suc}}{H_n}.
\end{split}
\end{displaymath}
For a large $n$, $P[W\ge2E[W]]$ is close to $0$. That is, the time for a node to discover all neighbors is very likely to be smaller than $2$ times of the expectation. Therefore, $W$ is also bounded by $O(n\ln n)$.

%%



%*****************








\subsection{Analysis of Gaussian Distribution}
\label{normal}
%background
Gaussian distribution is common adopted in wireless network. For instance, an intrusion detection application needs larger detection
probability around important entities\cite{wang2013gaussian}. We assume the positions of nodes obey 2D Gaussian distribution, and we present a theoretical proof that the discovery latency is not much larger than that of uniform distribution. 

Denote the approximate neighbors of node $u_i$ as set $N_i = \{u_{i1},u_{i2},...,u_{i\hat{n_i}}\}$. When nodes obey Gaussian distribution, the probability that node $u_i$ discovers a certain neighbor node $u_{ij}$ successfully in a time slot (denote as $P_{suc}$) can be formulated as:
$$
p_{suc} = (1-p_t^i)p_t^{ij}\prod_{ k=1}^{\hat{n_i}, k\neq j}(1-p_t^{ik})
$$

Denote $p_t^{imax} = \max_{1 \leq j \leq \hat{n_i}}\{p_t^{ij}\}$, $p_t^{imin} = \min_{1 \leq j \leq \hat{n_i}}\{p_t^{ij}\}$, for every $1 \leq j \leq n_i$, we have:
\begin{align*}
&(1-p_t^i)p_t^{imin}{(1-p_t^{imax})}^{\hat{n_i}-1} \\
\leq &(1-p_t^i)p_t^{ij}\prod_{ k=1}^{\hat{n_i}, k\neq j}(1-p_t^{ik}) \\
\leq &(1-p_t^i)p_t^{imax}{(1-p_t^{imin})}^{\hat{n_i}-1}
\end{align*}

Denote:
\begin{align*}
&P = (1-p_t^i)p_t^{imin}{(1-p_t^{imax})}^{\hat{n_i}-1}  \\
&Q = (1-p_t^i)p_t^{imax}{(1-p_t^{imin})}^{\hat{n_i}-1}
\end{align*}

We derive the expectation of $W_j$ for $1 \leq j \leq n_i$ as:
\begin{align*}
\frac{1}{\hat{n_i}Q} \quad \leq \quad &E[W_1] \quad \leq \quad \frac{1}{\hat{n_i}P} \\
\frac{1}{(\hat{n_i}-1)Q} \quad \leq \quad &E[W_2] \quad \leq \quad \frac{1}{(\hat{n_i}-1)P} \\
& ..........\\
\frac{1}{Q} \quad \leq \quad &E[W_{\hat{n_i}}] \quad \leq \quad \frac{1}{P}
\end{align*}

Combine the above equations to derive:
$$
\frac{1}{Q}H_n  \quad \leq \quad E[\sum_{j=1}^{\hat{n_i}}W_j]  \quad \leq \quad \frac{1}{P}H_n
$$

(???)Since the sensible neighbors are within a close distance of the node
compared to the total network area,
which implies the density function values are within the same order of magnitude.
Thus we can conclude that the expectation of the time latency in the normal
distributed networks are still $E[W]=O(nlnn)$.
Similarly the bounded latency can be proved to be $W=O(nlnn)$ in the same way as
uniform distribution.


