% Introduction
\section{Introduction}

%%%%%%

% Cite:
% Wireless networks			#gupta2000capacity
% Wireless sensor networks	#akyildiz2002wireless
% Volcanic investigation		#werner2006deploying

% Mobile campus networks 	#hernandez2005comparative
% Mobile gaming community	#cunningham2002optimizing
% Energy-efficient network	#jones2001survey
% Internet of things 			#qin2014software
% Connectivity  			#moscibroda2006complexity
% Density					#wang2015connectivity
% Unit disk graph model 		#clark1990unit
% CSMA/CA				#bianchi1996performance

% The cite of the existing algorithms are listed in related work

%%%%%%

% Paper Logic Flow


The popularity of Internet of Things (IoT) has turned people's attention back to wireless sensor networks\cite{akyildiz2002wireless}, with a wide range of applications such as volcanic investigation \cite{werner2006deploying}, seismic detection \cite{suzuki2007high}, agriculture monitoring \cite{wang2010l3sn}, etc. 

Neighbor discovery, in which a node tends to discover its nearby neighboring nodes before further processes like broadcasting and peer-to-peer communication, is a fundamental step to construct a wireless sensor network. In this paper, we study it in an energy-restricted large-scale scenario, where nodes are aware of energy consumption and  are multi-hop connected.

Unfortunately, despite extensive research, neighbor discovery in a large-scale network remains an open problem.
The existing algorithms can be classified into two categories: deterministic, and probabilistic.
In the deterministic algorithms \cite{dutta2008practical, kandhalu2010u, bakht2012searchlight, sun2014hello,  chen2015heterogeneous}, sensors take actions based on deterministic sequences.
However, most deterministic algorithms are designed only for two nodes and directly applied to a multi-node scenario. Probabilistic algorithms handle neighbor discovery in a clique of $n$ nodes\cite{mcglynn2001birthday, vasudevan2009neighbor, you2011aloha, song2014probabilistic} , i.e. every two nodes are neighbors, and utilize the global number $n$ to compute an optimal probability for action decisions. However, a large-scale network is not a clique and nodes are multi-hop connected. In addition, some existing algorithms, e.g. Birthday \cite{mcglynn2001birthday}, Aloha-like \cite{vasudevan2009neighbor}, do not consider energy consumption of the neighbor discovery process. In wireless sensor networks, sensors are powered by battery and stop working when energy is  depleted. Attempting for neighbor discovery consistently will be energy-consuming. 

Therefore, we look into the existing neighbor discovery algorithms and find that the key issue lies in the way to deal with collisions in the large-scale networks. %(collisions result in CSMA \cite{bianchi1996performance} to wait for more time).
This issue is due to three reasons.
First, transmission signals fade with distance and simultaneous transmissions will cause collisions among various nodes. Deterministic algorithms aiming at two nodes \cite{kandhalu2010u, chen2015heterogeneous} fail to reduce such collisions. Some beacon-based algorithms \cite{dutta2008practical, bakht2012searchlight, sun2014hello} do not have this problem but the time slot is 40 times larger \cite{kandhalu2010u} and still result in high latency. 
%There are many mature interference models to depict the communication collisions, and we choose the popular protocol model\cite{clark1990unit} to begin this research area.
Second, a large-scale network is not one-hop connected and a node can only discover its neighbors within its range. Probabilistic algorithms \cite{vasudevan2009neighbor, you2011aloha, song2014probabilistic} assuming a small clique network fail in estimating the number of neighbors, and thus can not reduce the collision effectively since the number of neighbors plays a vital role in how many collisions will occur at the same time. 
Third, nodes are powered by limited energy and they typically try to find neighbors only for a fraction of time. Energy consumption and neighbor discovery quality are a paradox in existing algorithms, since collisions cause great energy consumption and if taken energy consumption into account, neighbor discovery may become even ineffective.

We conduct experiments for fundamental observation and confirm this issue. We deployed $16$ EZ$240$ sensors \cite{huang2012easipled} and found existing algorithms reduced collisions either insufficiently or excessively, both resulting in a high latency.
%collisions caused by simultaneous transmissions result in the waste of time and energy. 
%We consider a large-scale network with 2000 nodes obeying a uniform distribution. 
As the number of neighbors increased, collisions of Hedis \cite{chen2015heterogeneous} happened as frequently as 10.1\% to 19.96\%, which called the CSMA \cite{bianchi1996performance} in MAC layer to wait for a random time. Hello \cite{sun2014hello} utilized beacon mechanism to avoid collisions but the time slot was 40 times larger and it resulted in 10 times higher latency.  Aloha-like \cite{you2011aloha} showed a high idle rate (none of the neighbors are transmitting) as 18.92\% of the time, which reduce the collisions excessively.  %and still results in a high latency. %This is the same with CSMA \cite{bianchi1996performance}, a general collision avoidance  technique in networks, the idle rate of which is XX\% by simulations. 
That is because of failing to deal with collisions, these algorithms waste time and energy, and cannot achieve low latency and energy efficiency at the same time.

Our insight idea is that, by estimating the expected number of neighbors of a node and synchronizing the time it turns on the radio with its neighbors, 
we can achieve both low-latency and energy-efficiency for neighbor discovery. 
We first take the distributions of nodes into consideration. As studied in \cite{wang2013gaussian}, nodes in a wireless sensor network are likely to follow a uniform or a Gaussian distribution for detecting aims. % (as shown in Fig. \ref{distribution}) 
According to the local density, a node can estimate the number of neighbors and calculate an optimal probability for action decisions. Then based on this, we propose Alano, %\footnote{Alano is the god of luck in Greek mythology.}
a nearly optimal probability based algorithm for a large-scale network. Finally, we involve the duty cycle mechanism, i.e. the fraction of time the radio is on (also called a sensor wakes up), and modify Alano by deterministic techniques to synchronize the wake-up time between neighbors. Specifically, if all nodes have the same (symmetric) duty cycle, such as a batch of sensors have a default duty cycle setting, we propose a Relaxed Difference Set based algorithm (called RDS-Alano); if nodes have different (asymmetric) duty cycles, such as a sensor adjusts the duty cycle by the remaining energy, we propose a Traversing Pointer based algorithm (called TP-Alano).

In simulations we have found Alano has $31.35\%$ to $ 85.25\%$ lower latency, higher discovery rate, and better scalability in large scale networks. %, and robustness, 
In comparison to the state-of-the-art algorithms \cite{you2011aloha, sun2014hello, chen2015heterogeneous, bakht2012searchlight}
Alano reaches nearly $100\%$ discovery rate within half time. 
When the number of nodes increases from 1000 to 9000, 
Alano shows 4.68 times to 6.51 times lower latency for neighbor discovery.

% Contribution summarize
The contributions of the paper are summarized as follows:
\begin{itemize}
\item[1)] We utilize distribution of nodes and propose Alano, a nearly optimal algorithm that achieves low-latency neighbor discovery for a large-scale network;
\item[2)] In an energy-restricted large-scale network, we propose RDS-Alano for symmetric nodes and TP-Alano for asymmetric nodes. Both algorithms achieve low latency for discovering neighbors and can prolong node's lifetime;
\item[3)] We conduct experiments for fundamental observation and extensive simulations for large-scale networks.  Alano achieves lower latency, higher discovery rate, and better scalability,
which promises a potential scalability of IoT in the future work.  %, and robustness compared with the state-of-the-art algorithms. 
\end{itemize}

%% Remaining structure
The remainder of the paper is organized as follows. The coming section highlights some related works and puts forward vital problems that the existing algorithms remain. The system model and basic definitions are introduced in Section \ref{sectionmodel}. We present Alano and show the method to combine nodes' distribution in Section \ref{PCN}. We propose two modified algorithms (RDS-Alano, TP-Alano) for an energy-restricted large-scale network for both symmetric and asymmetric nodes in Section \ref{EEN}. The extensive simulation results are shown in Section \ref{Evaluation} and we conclude the paper in Section \ref{Conclusion}.
