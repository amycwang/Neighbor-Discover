%\section{System Model and Problem Formulation}
\section{Preliminaries}
\label{sectionmodel}


In this section, we first give some notion definitions and introduce the collision detection mechanis. Then we formulate the Neighbor Discovery problem formally.  



\subsection{Sensor Node Model}


In the wireless sensor networks, the deployed sensor nodes keep their most time in sleep mode to avoid quick energy consumption.

For the sensor nodes, time is divided into slots. In each time slot, a node transform its state according to a pre-defined duty schedule.

\begin{definition}
\textbf{Duty schedule} is a pre-defined sequence $S=\{s^t\}_{0\leq t<T}$ of period $T$ and
$$ s^t=\left\{
\begin{aligned}
0  & & {sleep}\\
1  &  &{wake }\\
\end{aligned}
\right.
$$
\end{definition}

Each node repeats its duty schedule until finding all the neighbors.

\begin{definition}
\textbf{Duty circle} represents the fraction of one period T where a node turns its radio on. It can be formulated as:

$$\theta=\frac{|\{ 0\leq t<T : s^t =1\}}{T}.
$$
  
\end{definition}

When the sensor is in 

Notice that the neighbor discovery process is not bidirectional, which means a node discover one of its neighbor do







\subsection{Collision Detection Mechanism}









\subsection{Problem Definition}

Before we formulate the problem, we introduce three important
metrics of the performance of rendezvous methods. These metrics are
often used to measure the performance of rendezvous between two users.
%. Rendezvous is the process of establishing communication
%link between two neighbors, the metrics are mainly adopted to evaluate
%the performance between two users.
To begin with, we define
\textbf{Time to Rendezvous (TTR)} as:
\begin{definition}
Time to rendezvous is the number of time slots to select the
same available channel simultaneously.% once both users have started.
\end{definition}

This metric is commonly used in existing rendezvous
algorithms. Both \emph{maximum time to rendezvous (MTTR)} and
\emph{average time to rendezvous (ATTR)} are used to evaluate the
performance of the worst case and the average case
respectively\cite{Gu2013,Liu2012a,Yang2015,Ch2014,Chen2014}.

The second metric, \textbf{Rendezvous Degree (RD)}, is also widely
used. It evaluates the number of channels that the users can rendezvous
on.
\begin{definition}
Rendezvous degree is the percentage of possible rendezvous
channels relative to the total number of the jointly available channels.
\end{definition}

As interference exists for each available channel, communications
between the users could be inefficient if they rendezvous on a channel
with a lot of interference.
Therefore, we introduce a new metric \textbf{Rendezvous Interference (RI)} as:
\begin{definition}
Rendezvous interference is the product of the users' interference measurement at
the channel that they rendezvous on.
\end{definition}



\begin{problem}
For any channel set $C^*$ and interference set $I^*$, design the
channel hopping algorithm $f: t\mapsto C^*$, such that for $C_a, C_b
\subseteq C$, $C_a \bigcap C_b \neq \emptyset$, and any time drift
$\delta$, there exists $T_{\delta}$ and channel $c \in C_a \bigcap
C_b$ satisfying:
\begin{equation*}
f_{C_a, I_a}(T_{\delta}+\delta) = f_{C_b, I_b}(T_{\delta}) = c
\end{equation*}
\end{problem}

The algorithms for continuous rendezvous can be different from those
for the initial rendezvous. But the target is also to choose a quiet channel soon (small $RI$, small $TTR$ and large $RD$).


