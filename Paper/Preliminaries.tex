%\section{System Model and Problem Formulation}
\section{Preliminaries}
\label{sectionmodel}

In this section, we first give some notion definitions and introduce the collision detection mechanism. 
Then we formulate the Neighbor Discovery problem formally.  


\subsection{Sensor Node Model}

The wireless sensor network consists of a number of sensors distributed separately in a target area.
The deployed sensor nodes keep their most time in sleep pattern to avoid quick energy consumption 
and wake up timely to work on duty.

We assume that each node has a unique identifier $ID_i$. Time is divided into slots of equal length $t_0$, 
which is sufficient to finish communications. In each time slot, a node transform its pattern according to a pre-defined duty schedule.

\begin{definition}
\textbf{Duty schedule} is a pre-defined sequence $S=\{s^t\}_{0\leq t<T}$ of period $T$ and
$$ s^t=\left\{
\begin{aligned}
0  & & {sleep}\\
1  & & {wake}\\
\end{aligned}
\right.
$$
\end{definition}

 Each node construct its own duty schedule according to a specific strategy and repeats it
 until finding all the neighbors. Since the waking-up duration has a significant affect on the battery's lifetime, 
 duty circle is defined to restrict the energy consumption.

\begin{definition}
\textbf{Duty circle} represents the fraction of one period T where a node turns its radio on. It can be formulated as:

$$\theta=\frac{|\{ 0\leq t<T : s^t =1\}}{T}.
$$
  
\end{definition}

When a sensor wake up on a time slot, it can turn to either the transmitting state or listening state. 
\begin{itemize}
\item Transmitting state. A node turn to transmitting state will broadcast messages containing its own identify 
information to all its neighbors.
\item  Listening state. A node turn to listening state will monitor the frequency channel to collect its neighbors' information.
However collision will occur when two or more neighbor nodes transmit concurrently and thus no valid information will be gathered
\end{itemize}
Transiting between the states only costs little time, compared to one complete time slot.


\subsection{Collision Detection Mechanism}









\subsection{Problem Definition}

We consider a partially-connected sensor network model, 
where two nodes are neighbors if they locate within the radio range of each other. 
A  symmetric matrix is used to record the neighboring relations as:

$$ M_{i,j}=\left\{
\begin{aligned}
1  & & {Neighbor}\\
0  & & {Else}\\
\end{aligned}
\right.
$$

 each sensor follows its duty schedule to achieve neighbor discovery. 


 

Notice that the neighbor discovery process is not bidirectional, which means any pair of neighbors 
need to find each other separately. A sensor node $u_i$ find one of its neighbors $u_j$ can be formulated 
as $L(i,j)$. Then we define the discovery latency that node $u_i$ discovers all neighbors as:

\begin{definition}
\textbf{Discovery latency} of node $u_i$ is the time to discover all neighbors:
$$L(i) = \max_{M_{(i,j) = 1}} L (i,j).
$$
\end{definition}

Thus , the neighbor discovery problem can be formulated as:

\begin{problem}
Given a duty circle $\theta$, design a duty schedule and transiting strategy which optimizes $L(i)$ to the most extent. 
\end{problem}




