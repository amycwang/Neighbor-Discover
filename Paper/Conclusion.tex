% This section will talk about conclusion of the paper
\section{Conclusion}
\label{Conclusion}
In this paper, we initiate the study of neighbor discovery in an energy-restricted large-scale network.
To begin with, we propose Alano for a large-scale network where nodes' distribution are utilized to trigger the design of a node's transmitting probability. For different distributions, such as uniform distribution and normal distribution, we show that Alano achieves nearly optimal discovery latency. Then, we propose two modification methods for a energy-restricted network on the basis of different duty cycle mechanisms: Relaxed Different Set based Alano (RDS-Alano) for symmetric nodes and Traversing Pointer based Alano (TP-Alano) for asymmetric nodes. We conduct extensive simulations to compare Alano with the state-of-the-art algorithms, the results show that Alano achieves better performance regarding discovery latency, discovery rate (quality), scalability and robustness.  
