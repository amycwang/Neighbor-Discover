% As a general rule, do not put math, special symbols or citations
% in the abstract
\begin{abstract}

Neighbor discovery is a fundamental step in constructing wireless networks and many elegant protocols have been proposed to minimize discovery latency. However, few of them can be applied to an energy-restricted large-scale network, which is more appealing and promising due to the development of intelligent devices. A mobile campus network or an intelligent vehicular network are typical examples.
In an energy-restricted large-scale network, a node has limited power supply and it can only discover a part of nodes that are within distance range; in addition, the discovery process may fail if much communication exists on the wireless channel. These factors make neighbor discovery a challenging task in establishing the networks.

In this paper, we propose Alano, the first nearly optimal algorithm for a large-scale network on the basis of nodes' distributions.
When nodes have same energy constraints, we modify Alano by Relaxed Difference Set (RDS) (denote as RDS-Alano); while we present a Traversing Pointer (TP) based Alano (denote as TP-Alano) when the energy constraints are different. We compare Alano with the state-of-the-art protocols through extensive evaluations, and the results show that Alano achieves about $30\%$ lower discovery latency and it has higher performance regarding quality (discovery rate), scalability and robustness.
\end{abstract}

% no keywords